

\begin{center}
\large \bf{Abstract}
\end{center}
\\
\textbf{Purpose} :- This study investigates the key factors influencing household expenditure patterns in Uttar Pradesh, India. It examines how income, demographic variables, and asset ownership jointly affect consumption behavior across rural and urban households.
\\
\\
\textbf{Design/Methodology/Approach} :- The analysis utilizes primary survey data collected from 320 households in four districts. An Ordinary Least Squares (OLS) regression framework was employed to model the relationship between total monthly household expenditure and multiple socio-economic predictors. The model’s validity was tested using cross-validation and diagnostic checks for multicollinearity, heteroscedasticity, and residual normality.
\\
\\
\textbf{Findings} :- Results indicate that household income strongly predicts total expenditure, aligning with Engel’s law. Family size and the education level of the household head were also found to exert moderate influence, whereas asset ownership variables such as televisions, bikes, and cars had limited statistical significance. Regional and district differences did not significantly affect consumption patterns, suggesting homogeneity in household behavior within the sampled areas.
\\
\\
\textbf{Research limitations/implications } :- As a cross-sectional study based on self-reported data, findings are subject to recall bias and cannot establish causal relationships. Future studies could enhance generalizability by using longitudinal designs and larger, more diverse samples.
\\
\\
\textbf{Practical implications} :- Insights from this study can inform policymakers and development planners in designing welfare schemes and income-support initiatives tailored to the specific socio-economic context of Uttar Pradesh.
\\
\\
\textbf{Originality/value} :- This research provides one of the few region-specific, micro-level analyses of household expenditure determinants in Uttar Pradesh. It bridges a key gap in the literature by jointly modeling income, demographics, and asset variables within a single empirical framework.
\\
\\
\textbf{Keywords:} Household consumption, Uttar Pradesh, Income determinants, Demographics, Asset ownership, OLS regression
% \cleardoublepage
